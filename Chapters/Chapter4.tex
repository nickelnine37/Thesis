\chapter{Regression and Reconstruction on Cartesian Product Graphs} 

\label{chap:reg_and_rec} 

\lhead{Chapter 4. \emph{Regression and Reconstruction on Cartesian Product Graphs}} 



\section{Introduction}

\label{sec:reg_and_rec_intro}

In this chapter, we turn our attention to the topic of signal processing on \textit{Cartesian product graphs}. This special class of graph finds applications in numerous areas, such as video, hyper-spectral image processing and network time series problems. However, the Cartesian product is not the only way to define a product between two graphs and, in general, there may be many consistent ways to define it. Other prominent examples include the strong, direct and lexicographic products \citep{Imrich2000}. 

In the general case, we consider two undirected graphs $\mathcal{G}_N = (\mathcal{V}_N, \mathcal{E}_N)$ and $\mathcal{G}_T = (\mathcal{V}_T, \mathcal{E}_T)$ with vertex sets given by $\mathcal{V}_N = \{n \in \mathbb{N} \, | \, n \leq N \}$ and $\mathcal{V}_T = \{t \in \mathbb{N} \, | \, t \leq T \}$ respectively. (In the present context we do not regard zero to be an element of the natural numbers). A new graph $\mathcal{G}$ can be constructed by taking the product between $\mathcal{G}_N$ and $\mathcal{G}_T$. This can be generically written as follows. 

\begin{equation}
    \mathcal{G} = \mathcal{G}_N \, \diamond \, \mathcal{G}_T = (\mathcal{V}, \, \mathcal{E})
\end{equation}

For all definitions of a graph product, the new vertex set $\mathcal{V}$ is given by the Cartesian product of the vertex sets of the factor graphs, that is

\begin{equation}
    \mathcal{V} = \mathcal{V}_N \times \mathcal{V}_T = \{(n, \, t) \in \mathbb{N}^2 \, | \, n \leq N \; \text{and} \; t \leq T \}
\end{equation}

Each consistent rule for constructing the new edge set $\mathcal{E}$ corresponds to a different definition of a graph product. For the Cartesian product, which is commonly denoted as $\mathcal{G}_N \, \square \, \mathcal{G}_T$, the new edge set is constructed as follows. 

\begin{equation}
    \mathcal{E} = \Big\{\, \big[(n, \, t), \, (n',  t') \big] \; \Big| \; \big(n = n' \; \text{and} \; [t, t'] \in \mathcal{E}_T \big) \; \text{or} \; \big(t = t' \; \text{and} \; [n, n'] \in \mathcal{E}_N\big)  \Big\}
\end{equation}
     

This can be described more succinctly by considering the respective adjacency matrices of the factor graphs. If $\mathcal{G}_N$ and $\mathcal{G}_T$ have adjacency matrices given by $\A_N$ and $\A_T$, then the adjacency matrix of their Cartesian product is given by the Kronecker sum of their respective adjacency matrices \citep{Fiedler1973}. 

\begin{equation}
    \A = \A_N \oplus \A_T 
\end{equation}

Note that while, given this definition, it may seem that the Cartesian product of graphs is non-commutative since, in general, $\A_N \oplus \A_T  \neq \A_T \oplus \A_N $, the graphs $\mathcal{G}_N \, \square \, \mathcal{G}_T$ and $\mathcal{G}_T \, \square \, \mathcal{G}_N$ are isomorphically identical \citep{Imrich2000}. The only key difference arises from the ordering of the node sets. 

% The direct and strong graph products have a similarly succinct representation in terms of the factor adjacency matrices. Specifically, 


% \begin{align}
%     \text{Direct:} & \; \A = \A_N \otimes \A_T \\ \text{Strong:} & \; \A = \A_N \otimes \A_T + \A_N \oplus \A_T
% \end{align}

\section{Graph Signal Reconstruction on Cartesian Product Graphs}

\label{sec:gsr_cpg}

One task that is of particular interest 

\subsection{A stationary iterative method}

Hello

\subsection{A conjugate gradient method}

Hello

\subsection{Convergence properties}

Hello

\subsection{Image processing experiments}

Hello



\section{Kernel Graph Regression with Unrestricted Missing Data Patterns}

\label{sec:kgr_mdp}

Hello

\subsection{Cartesian product graphs and KGR}

Hello

\subsection{Convergence properties}

Hello


\section{Regression with Network Cohesion}

\label{sec:rnc_mdp}

Hello

\subsection{Regression with node-level covariates}

Hello

\subsection{Convergence properties}

Hello


\section{Multi-Dimensional Cartesian Product Graphs}

\label{sec:nd_gsp}

Hello

\subsection{Fast computation with \textit{d}-dimensional Kronecker products}

Hello

\subsection{Signal reconstruction}

Hello

\subsection{Kernel Graph Regression}

Hello

\subsection{Regression with Network Cohesion}


