\chapter{Literature Review and Thesis Outline} % Main chapter title

\label{chap:outline} % Change X to a consecutive number; for referencing this chapter elsewhere, use \ref{ChapterX}

\lhead{Chapter 2. \emph{Outline and Fundamentals}} % Change X to a consecutive number; this is for the header on each page - perhaps a shortened title


\section{Graph Signal Processing}


\subsection{A broad overview of the field}


\subsection{The graph Laplacian}

\cite{LeMagoarou2016}


\subsection{Graph filters}

\subsubsection{Graph Kernels}


\begin{table}[b]
    \renewcommand{\arraystretch}{1.7}
    \small
    \begin{center}
    \begin{tabular}{|l|c|}
    \hline
    \textbf{Filter}   & $g(\lambda; \,\beta)$    \\ 
    \hline
    1-hop random walk & $(1 + \beta \lambda)^{-1}$ \\
    \hline
    Diffusion         & $\exp(-\beta \lambda)$       \\
    \hline
    ReLu              & $\max (1 - \beta \lambda, 0)$  \\
    \hline
    Sigmoid           & $2 \big( 1 + \exp(\beta \lambda)\big)^{-1}$ \\
    \hline
    Bandlimited       & $1, \,\text{if} \; \beta \lambda \leq 1 \; \text{else} \; 0$   \\
    \hline
    \end{tabular}
    \end{center}
    \caption{Isotropic graph filter functions}
    \label{tab:iso_filters}
    \end{table}



\section{Regression and Reconstruction}


\subsection{Graph Signal Reconstruction}


\subsection{Kernel Graph Regression}

\subsubsection{Gaussian Processes on Graphs}


\subsection{Regression with Network Cohesion}


\section{GSP on higher order graphs}

Multiway data processing 

\cite{Smilde2004}
\cite{Kroonenberg2008}


\cite{Ji2019}

\cite{Cammoun2009}


\subsection{Multi-Layer Graph Signal Processing}

\cite{Zhang2022} describe M-GSP 

\cite{Zhang2018} extend M-GSP to multiple layers with different number of nodes by adding fake nodes in where they are missing from layers. 
 

\subsection{Multi-Way Graph Signal Processing}


